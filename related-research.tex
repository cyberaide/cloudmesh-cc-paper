\FILE{related-research.tex}

\section{Related Research}

\section{HPC Batch Queueing Systems}

\subsection{Scientific and Grid Computing Workflows}

\subsection{Cloud Workflows}

\subsection{REST interfaces}

\begin{table*}[htb]

\caption{TBD}

\resizebox{1.0\textwidth}{!}{
{\footnotesize
\begin{tabular}{|p{3cm}p{12cm}|}
\hline
{\bf Name} & {\bf Description} \\
\hline
\hline
\multicolumn{2}{|l|}{\em\bf HPC Batch Workflow} \\
\hline

LSF \cite{www-lsf} & Batch Queue Manager. \\
\hline

SLURM \cite{www-slurm} & Batch queue workflow manager. \\
\hline


Cloudmesh  \cite{www-cloudmesh-manual} & Suite 
of Python code for cloud computing. \\
\hline

Airflow \cite{www-airflow} & ``Airflow is a platform created by the community to programmatically author, schedule and monitor workflows.'' \\
\hline


Loosely coupled Metacomputer \cite{las-1996-thesis,las-1999-loosely} & Early work on workflow management system comnnecting multiple super computers. \\
\hline

Karajan, cog kit  \cite{las07-workflow} & Sophisticated Workflow management system with Dag and loops for Grid computing and loosely coupled compute resources thorugh services. \\
\hline

Snakemake  \cite{www-snakemake} & ``The Snakemake workflow management system is a tool to create reproducible and scalable data analyses. Workflows are described via a human readable, Python based language. They can be seamlessly scaled to server, cluster, grid and cloud environments, without the need to modify the workflow definition. Finally, Snakemake workflows can entail a description of required software, which will be automatically deployed to any execution environment.'' \\
\hline

Gridant  \cite{las-2004-gridant} & 
Client based workflow management toolkit to orchestrate complex workflows on the fly without substantial help from the service providers. It integrates on premisse and Grid resources. \\
\hline

Keppler  \cite{www-kepler} & ``Kepler is designed to help scientists, analysts, and computer programmers create, execute, and share models and analyses across a broad range of scientific and engineering disciplines.'' \\
\hline

Pegasus  \cite{www-pegasus} &  A DAG based workflow management tool for scientific workflow.  \\
\hline

Swift \cite{las-2007-swift} & A language for distributed parallel scripting.\\
\hline

Parsl \cite{www-parsl} & A language for distributed parallel scripting. \\
\hline

Radical Pilot \cite{arxiv-radical-pilot} & Pilot system that enables scalable workflows \\ 
\hline

Gcloud workflow \cite{www-gcloud} & Google Cloud documentation for creating a workflow with the command line interface. \\
\hline

Artifact Visualization \cite{www-artifact} & Argo Workflows documentation for kubernetes which shows how to visualize artifacts. \\
\hline

REST Capabilities for RESTful Workflows \cite{www-rest-workflows} & Article that demonstrates various examples of REST APIs, such as KNIME. \\
\hline

Business Automation with REST \cite{www-business-rest-ibm} & IBM REST API documentation for its Business Automation Workflows. \\
\hline

Azure REST \cite{www-azure-rest} & Microsoft Azure REST API documentation to interface with Azure Logic Apps. \\
\hline

GitHub REST Cancel a Workflow Run \cite{www-github-rest-cancel} & GitHub REST API documentation to cancel a GitHub Workflow. \\
\hline

ArcGIS Rest \cite{www-arcgis-rest} & ArcGIS documentation for its REST API to interface with the ArcGIS Server. \\
\hline

Azure Enterprise Workflow \cite{www-azure-enterprise-workflow} & Article showcasing how to use Enterprise Workflows with Azure Logic Apps. \\
\hline

WSDL \cite{www-wsdl} & Microsoft Azure schema reference for the Workflow Definition Language in Azure Logic Apps. \\
\hline

AWS Workflow \cite{www-aws-workflow} & Amazon Web Services documentation definition of a workflow.\\
\hline

AWS SWF \cite{www-aws-swf} & Amazon Web Services documentation for managing simple workflows with tasks. \\
\hline

AWS step functions \cite{www-aws-stepfunctions} & Amazon Web Services documentation for visual workflow service. \\
\hline

AWS HPC  \cite{www-aws-batch-workflow} & Amazon Web Services documentation for setting workflow dependencies using batch processing in HPC on the cloud. \\
\hline

azure batch: \cite{www-azure-batch} & Microsoft Azure documentation for running batch service workflows. \\
\hline

\end{tabular}
}
}
\end{table*}
