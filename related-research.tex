\FILE{related-research.tex}

\section{Related Research}

Many different workflow systems have been developed in the past.  It
is out of scope of this document to present a complete overview of all
the different stystems. Instead we compare some selected features of
the systems and identify features that are important for us. The
following features are important.

\begin{description}

\item[Batch.] Long running jobs on HPC systems are coordinated through Batch services.

\item[Resource Reservation.] In some cases access to batch queuses on HPC systems take a long time in case of many different tasks it is sometimes useful to researce a number of batch nodes and run many short running programs on them if the need arises. THis haowever can often be replacesd just with properly coordinated workflows using a batch system. In fact frameworks using such reservation internally implement them using the batch system.

\item[Computational Grids.] Grids provide a service level access to distributed HPC computing resources. However Grids popular a decade ago are no longer the predominantly deployed and focus has shifted to CLoud computing. 

\item[Cloud.] Computing resourcesa are these days also available in the cloud as HPC, batch, and compute services, including specialized SaaS offerings that allow to integrate analytics finctions into workflows.

\item[REST.] The predominant specification for cloud services uses the REST framework relying on a stateles model. This contrasts the WSRF model that uses statefull services.

\item[Specification.] Some work has focused on the spcification of the workflows rather.

\item[GUI.] Some foremworks provide extensive GUIs.

\end{description}

In Table \ref{tab:workflow-table} we provide a number of examples for the various
features we found in workflow systems.


\begin{table*}[htb]

\caption{Example of workflow frameworks with selected features.}
\label{tab:workflow-table}

\resizebox{1.0\textwidth}{!}{
{\footnotesize
\begin{tabular}{|p{3cm}|p{3cm}|p{10cm}|}
\hline
{\bf Name} & {\bf Selected \n Features} & {\bf Description} \\
\hline
\hline

LSF \cite{www-lsf} & Batch, HPC & Batch Queue Manager. \\
\hline

SLURM \cite{www-slurm} & Batch, HPC & Batch queue workflow manager. \\
\hline

Cloudmesh  \cite{www-cloudmesh-manual} & HPC, Cloud, on-Premisse & Suite 
of Python code for cloud computing. \\
\hline

Airflow \cite{www-airflow} & HPC, Cloud, Remote Tasks & ``Airflow is a platform created by the community to programmatically author, schedule and monitor workflows.'' \\
\hline


Loosely coupled Metacomputer \cite{las-1996-thesis,las-1999-loosely} &
                                                                       HPC, on-Premisse
             & Early work on workflow management system comnnecting multiple super computers. \\
\hline

Karajan CoGkit  \cite{las07-workflow} & Service, Langauage, HPC, on-Premisse & Sophisticated Workflow management system with Dag and loops for Grid computing and loosely coupled compute resources thorugh services. \\
\hline

Snakemake  \cite{www-snakemake} & Language & ``The Snakemake workflow management system is a tool to create reproducible and scalable data analyses. Workflows are described via a human readable, Python based language. They can be seamlessly scaled to server, cluster, grid and cloud environments, without the need to modify the workflow definition. Finally, Snakemake workflows can entail a description of required software, which will be automatically deployed to any execution environment.'' \\
\hline

Gridant  \cite{las-2004-gridant} & Language & 
Client based workflow management toolkit to orchestrate complex workflows on the fly without substantial help from the service providers. It integrates on premisse and Grid resources. \\
\hline

Keppler  \cite{www-kepler} & Service, GUI & ``Kepler is designed to help scientists, analysts, and computer programmers create, execute, and share models and analyses across a broad range of scientific and engineering disciplines.'' \\
\hline

Pegasus  \cite{www-pegasus} & DAG, Service, HPC, Coud &  A DAG based workflow management tool for scientific workflow.  \\
\hline

Swift \cite{las-2007-swift} & Language, Many task & A language for distributed parallel scripting.\\
\hline

Parsl \cite{www-parsl} & Language, Many task & A language for distributed parallel scripting. \\
\hline

Radical Pilot \cite{arxiv-radical-pilot} & Many task & Pilot system that enables scalable workflows \\ 
\hline

Gcloud workflow \cite{www-gcloud} & Cloud & Google Cloud documentation for creating a workflow with the command line interface. \\
\hline

Azure REST \cite{www-azure-rest} & Cloud & Microsoft Azure REST API documentation to interface with Azure Logic Apps. \\
\hline

GitHub REST Cancel a Workflow Run \cite{www-github-rest-cancel}
& GitHub Actions
& GitHub REST API documentation to cancel a GitHub Workflow. \\
\hline

ArcGIS Rest \cite{www-arcgis-rest} & & ArcGIS documentation for its REST API to interface with the ArcGIS Server. \\
\hline

Azure Enterprise Workflow \cite{www-azure-enterprise-workflow} & & Article showcasing how to use Enterprise Workflows with Azure Logic Apps. \\
\hline

WSDL \cite{www-wsdl} & Specification & Microsoft Azure schema reference for the Workflow Definition Language in Azure Logic Apps. \\
  \hline

  WSRF \cite{www-wsrf} & Specification & OASIS Web Services Resource Framework\\

AWS Workflow \cite{www-aws-workflow} & & Amazon Web Services documentation definition of a workflow.\\
\hline

AWS SWF \cite{www-aws-swf} & & Amazon Web Services documentation for managing simple workflows with tasks. \\
\hline

AWS Step Functions \cite{www-aws-stepfunctions} & & Amazon Web Services documentation for visual workflow service. \\
\hline

AWS HPC  \cite{www-aws-batch-workflow} & Cloud, HPC & Amazon Web Services documentation for setting workflow dependencies using batch processing in HPC on the cloud. \\
\hline

Azure Batch \cite{www-azure-batch} & Cloud, Batch & Microsoft Azure documentation for running batch service workflows. \\
  \hline

 Business Automation with REST \cite{www-business-rest-ibm} & Buisiness
                                        Process& IBM REST API documentation for its Business Automation Workflows. \\
\hline

  
% --  Artifact Visualization \cite{www-artifact} & & Argo Workflows documentation for kubernetes which shows how to visualize artifacts. \\
% \hline

% -- REST Capabilities for RESTful Workflows \cite{www-rest-workflows} & & Article that demonstrates various examples of REST APIs, such as KNIME. \\
% \hline

Infogram \cite{las-02-infogram} & Service & A Peer-to-Peer Information and Job Submission Service.\\
\hline
\end{tabular}
}
}
\end{table*}
