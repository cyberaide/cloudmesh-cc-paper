\FILE{conclusion.tex}

\section{Conclusion}

We have designed and implemented a Hybrid Reusable Computational
Analytics Workflow Management with the help of the cloudmesh
component framework. The component added focuses on the management of
workflows for computational analytics tasks and jobs. The tasks can be
executed on remote resources via ssh and even access queuing systems
such as SLURM. In addition, we can integrate the current computer on
which the workflow is running. This can include operating systems such
as Linux, macOS, Windows, and even Windows Subsystem for Linux. Through
cloudmesh access to a command line and a command shell is provided. A
simple API and a REST interface are provided. The framework has also an
elementary Web browser interface that allows visualizing the
execution of the workflow.  It is important to know that the workflow
can be started on remote resources and is running completely
independently from the client tool once a task is started. This allows a
``stateless'' model that can synchronize with the remotely started jobs
on demand. Hence the framework is self-recovering in case of network
interruptions or power failure. Due to our experiences with real (and
many) infrastructure failures at the authors' locations, the
availability of such a workflow-guided system was
beneficial. Furthermore, the code developed is rather small and in
contrast to other systems is less complex. Hence it is suitable for
educational aspects as it is used for Master's level and undergraduate
research projects. The project has also been practically utilized while
generating benchmarks for the MLcommons Science Working Group
showcasing real-world applicability beyond a student research project.
