\hypertarget{installation}{%
\section{Installation}\label{installation}}

Installation is relatively simple. We leverage the cloudmesh-installer
to locally install the cloudmesh suite of repositories.

\begin{Shaded}
\begin{Highlighting}[]
\FunctionTok{mkdir}\NormalTok{ \textasciitilde{}/cm}
\BuiltInTok{cd}\NormalTok{ \textasciitilde{}/cm}
\ExtensionTok{pip}\NormalTok{ install cloudmesh{-}installer }\AttributeTok{{-}U}
\ExtensionTok{cloudmesh{-}installer}\NormalTok{ get cc}
\end{Highlighting}
\end{Shaded}

\hypertarget{windows}{%
\subsection{Windows}\label{windows}}

Git Bash and Graphviz must be installed. The user can use Chocolatey run
as an administrator for convenience:

\begin{Shaded}
\begin{Highlighting}[]
\ExtensionTok{choco}\NormalTok{ install git.install }\AttributeTok{{-}{-}params} \StringTok{"/GitAndUnixToolsOnPath }\DataTypeTok{\textbackslash{}}
\StringTok{        /Editor:Nano /PseudoConsoleSupport /NoAutoCrlf"} \AttributeTok{{-}y}
\ExtensionTok{choco}\NormalTok{ install graphviz }\AttributeTok{{-}y}
\end{Highlighting}
\end{Shaded}

\hypertarget{macos}{%
\subsection{macOS}\label{macos}}

Graphviz must be installed. The user can use Homebrew for convenience:

\begin{Shaded}
\begin{Highlighting}[]
\ExtensionTok{brew}\NormalTok{ install graphviz}
\end{Highlighting}
\end{Shaded}

\hypertarget{linux}{%
\subsection{Linux}\label{linux}}

Graphviz must be installed. The user can use apt for convenience:

\begin{Shaded}
\begin{Highlighting}[]
\FunctionTok{sudo}\NormalTok{ apt install graphviz }\AttributeTok{{-}y}
\end{Highlighting}
\end{Shaded}
