\FILE{requirements.tex}

\section{Requirements}

For the design of the framework we have the following requirements.

\begin{description}

\item[Simplicity.] The design must e simpla and the code base must be small. THis helps the future maintenance of the code, but also allows the code to become part of eductaional oppertunities such as in Research experience for undergraduates (REU), capstone project, and class projects. The challange here is that many other systems are too big to be used in introductory undergraduate and mastares activities. 
However the framework should be capable enough to support also research projects.

\item[Workflow specification.]  The specification of the workflow must be simple. From the past we have learned that the introduction of programming controll components such as loops, and conditions, in addition to DAG's is essential to provide maximum flexibility. 

\item[Workflow Monitoring.] The workflow framework must be able to monitor the progress of individual jobs as well as the overall progress of the workflow. 

\item[Workflow Interfaces.] To specify and interface with the workflow we must provide several interface layers. This includes, specification through a YAML file, interfacing  through REST calles, interfacing through a python API, and interfaceing with a very simple GUI component. For the workflow monitoring with the GUI we must be able to easily define custom text to reflect user designable monitoring labels.

\item[Langaugae Independence.] As we want to make the framework integratable in other frameworks we need a simple mechanism to provide either API or interface portability. To keep the code base small a REST API seems very well suited.

\item[OpenAPI.] To further strengthen usability the framework must have an OpenAPI interface. THis allows the integration via 3rd party tools into other frameworks and languages if desired.

\item[Hybrid multicloud Providers.] The service must be able to be deployable on an on-premisse local computer or various cloud providers.

\item[Generalized Job Interface.] The framework must be able to interface 
with a wide variety of computation services. This includes the support
of ssh, batch queues such as SLURM and LSF, local compute resources
including shell scripts as well as support for WSL.

\item[Support of various Operating Systems.] The framework must be runnable on various client operating systems including Linux, macOS, Windows.

\item[State Management.] The state management must be recoverable and the client must be able to completely recover from a network failure. As such the state will be queried on demand. This allows one to deploy the framework on a laptop, start the workflow, close or shutdown the laptop and at a alater time open the laptop while the workflow framework can be refreshed with the latest state.

\end{description}
