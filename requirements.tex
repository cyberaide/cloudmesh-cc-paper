\FILE{requirements.tex}

\section{Requirements}

For the design of the framework, we have the following requirements.

\begin{description}

\item[Simplicity.] The design must be simple and the code base must be small.
This helps the future maintenance of the code, but also allows the code
to become part of educational opportunities such as in a Research
Experience for Undergraduates (REU), capstone project, and class
projects. The challenge here is that many other systems are too big to
be used in introductory undergraduate and masters activities.
However, the framework should be capable enough to also support research projects.

\item[Workflow specification.] The specification of the workflow must be simple.
From the past we have learned that the introduction of programming control
components such as loops, and conditions, in addition to DAGs, is
essential to provide maximum flexibility.

\item[Workflow Monitoring.] The workflow framework must be able to monitor the
progress of individual jobs as well as the overall progress of the workflow.

\item[Workflow Interfaces.] To specify and interface with the workflow, we
must provide several interface layers. This includes specification
through a YAML file, interfacing through REST calls, interfacing
through a Python API, and interfacing with a very simple GUI component.
For the workflow monitoring with the GUI, we must be able to easily
define custom text to reflect user designable monitoring labels.

\item[Language Independence.] As we want to make the framework integrable in
other frameworks, we need a simple mechanism to provide either API or
interface portability. To keep the code base small, a REST API seems
very well-suited.

\item[OpenAPI.] To further strengthen usability, the framework must have an
OpenAPI interface. This allows the integration via 3rd party tools into
other frameworks and languages, if desired.

\item[Hybrid multicloud Providers.] The service must be able to be deployable
on an on-premise local computer or various cloud providers.

\item[Generalized Job Interface.] The framework must be able to interface 
with a wide variety of computation services. This includes the support
of ssh, batch queues such as Slurm and LSF, and local compute resources
including shell scripts, as well as support for WSL.

\item[Support of various Operating Systems.] The framework must be runnable on
various client operating systems including Linux, macOS, Windows.

\item[State Management.] The state management must be recoverable and the
client must be able to completely recover from a network failure.
As such, the state will be queried on demand. This allows one to deploy
the framework on a laptop, start the workflow, close or shutdown the
laptop, and at a later time open the laptop while the workflow framework
can be refreshed with the latest state.

\end{description}
