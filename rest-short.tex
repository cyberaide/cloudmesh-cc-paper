\FILE{rest-short.tex}

\section{REST}\label{rest}

Cloudmesh cc can be interfaced with Representational State Transfer
(REST) by using Curl commands. REST is advantageous as it can be used
with programming languages other than Python.

To use this feature, ensure that the FastAPI server is online with the
following commands:

\begin{minted}[breaklines]{bash}
$ cd ~/cm/cloudmesh-cc
$ cms cc start
\end{minted}

\subsection{List Workflows Available on Local
Computer}\label{list-workflows-available-on-local-computer}

The following command returns a dict that lists each available workflow.

\begin{minted}[breaklines]{bash}
curl -X 'GET' 'http://127.0.0.1:8000/workflows' -H 'accept: application/json'
\end{minted}

The rest of the Curl examples will use \mintinline{bash}|workflow-example| as
an example. To use the Curl commands, replace \mintinline{bash}|workflow-example| 
with the name of the workflow you wish to run.

\subsection{Workflow Management}\label{retrieve-a-workflow}

We can use HTTP methods to manage a preexisting workflow. We can leverage
\mintinline{bash}|GET| and \mintinline{bash}|DELETE| to retrieve and
remove a workflow, respectively.

The following command retrieves a workflow and its configuration, such
as the directory where it resides, the jobs that it contains, the colors
of the statuses, and more.

To use it, replace \texttt{workflow-example} with the workflow you wish
to retrieve.

\begin{minted}[breaklines]{bash}
$ curl -X 'GET' 'http://127.0.0.1:8000/workflow/workflow-example' -H 'accept: application/json'
\end{minted}

To delete the workflow, use \mintinline{bash}|DELETE| instead of \mintinline{bash}|GET|.

\subsection{Upload a Workflow}\label{upload-a-workflow}

There are three different ways to upload a new workflow to cloudmesh-cc.
The most convenient way is to specify a directory that contains the
yaml configuration file and scripts with the following command:

\begin{minted}[breaklines]{bash}
$ curl -X 'POST' 'http://127.0.0.1:8000/workflow?directory=~/cm/cloudmesh-cc/tests/workflow-example' -H 'accept: application/json' -d ''
\end{minted}
\smallskip

The workflow is then ready to run via the REST GUI Web Interface,
via the Python interface, or other interfaces.

\subsection{Run a Workflow}\label{run-a-workflow}

The following command runs a workflow.

\begin{minted}[breaklines]{bash}
$ curl -X 'GET' 'http://127.0.0.1:8000/workflow/run/workflow-example?show=True' -H 'accept: application/json'
\end{minted}

Additionally, to hide the graph of the workflow as it is run,
replace \mintinline{bash}|show=True| with \mintinline{bash}|show=False|.

