\hypertarget{cloudmask-workflow}{%
\section{Cloudmask Workflow}\label{cloudmask-workflow}}

Cloudmesh cc comes with an example workflow that runs Cloudmask, which
is a program that develops a model to classify sections of satellite
images. Information regarding Cloudmask can be found at
\url{https://github.com/laszewsk/mlcommons/tree/main/benchmarks/cloudmask\#readme}

\hypertarget{running-the-cloudmask-workflow-on-rivanna}{%
\subsection{Running the Cloudmask Workflow on
Rivanna}\label{running-the-cloudmask-workflow-on-rivanna}}

To execute the workflow on UVA's HPC supercomputer, Rivanna, first
ensure that your UVA Computing ID is set with the following command,
replacing the X's with your ID:

\begin{Shaded}
\begin{Highlighting}[]
\ExtensionTok{me@mycomputer}\NormalTok{ $ cms set username=XXXXXX}
\end{Highlighting}
\end{Shaded}

Then, connect to the UVA Anywhere VPN and then download the satellite
data into your
\href{https://www.rc.virginia.edu/userinfo/storage/non-sensitive-data/\#scratch}{scratch}
directory on Rivanna:

\begin{Shaded}
\begin{Highlighting}[]
\ExtensionTok{me@mycomputer}\NormalTok{ $ cms vpn connect}
\ExtensionTok{me@mycomputer}\NormalTok{ $ ssh rivanna}
\ExtensionTok{rivanna}\NormalTok{ $ cd /scratch/}\VariableTok{$USER}
\ExtensionTok{rivanna}\NormalTok{ $ git clone https://github.com/laszewsk/mlcommons.git}
\ExtensionTok{rivanna}\NormalTok{ $ cd mlcommons/benchmarks/cloudmask/target/rivanna}
\ExtensionTok{rivanna}\NormalTok{ $ make data}
\CommentTok{\# downloading the data will take a while.}
\ExtensionTok{rivanna}\NormalTok{ $ exit}
\end{Highlighting}
\end{Shaded}

Next, clone the \texttt{mlcommons} repository on your local machine and
run the workflow:

\begin{Shaded}
\begin{Highlighting}[]
\ExtensionTok{me@mycomputer}\NormalTok{ $ cd \textasciitilde{}/cm}
\ExtensionTok{me@mycomputer}\NormalTok{ $ git clone https://github.com/laszewsk/mlcommons.git}
\ExtensionTok{me@mycomputer}\NormalTok{ $ cd mlcommons}
\ExtensionTok{me@mycomputer}\NormalTok{ $ pytest }\AttributeTok{{-}v} \AttributeTok{{-}x} \AttributeTok{{-}{-}capture}\OperatorTok{=}\NormalTok{no benchmarks/cloudmask/target/rivanna/run\_cloudmask\_workflow.py}
\end{Highlighting}
\end{Shaded}

The workflow iterates through the five GPUs available on Rivanna---
A100, V100, P100, RTX2080, and K80--- and runs the program three times
on each GPU. Each run trains the model with 10, 30, and 50 epochs for
benchmarking.

Upon completing a run, the logs and benchmarks of the program can be
found in the target folder:

\begin{Shaded}
\begin{Highlighting}[]
\ExtensionTok{me@mycomputer}\NormalTok{ $ ssh rivanna}
\ExtensionTok{rivanna}\NormalTok{ $ cd /scratch/}\VariableTok{$USER}\NormalTok{/mlcommons/benchmarks/cloudmask/target}
\end{Highlighting}
\end{Shaded}

Additionally, the generated \texttt{.h5} model file can be found in the
home directory:

\begin{Shaded}
\begin{Highlighting}[]
\ExtensionTok{rivanna}\NormalTok{ $ cd \textasciitilde{}/sciml\_bench/outputs/slstr\_cloud/}
\end{Highlighting}
\end{Shaded}

The program may take a while to run if the resources on Rivanna are
being used by other jobs.
