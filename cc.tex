\FILE{cc.tex}

\section{Introduction}

In this section we provide an introduction to or work while 
moving forward to motivate a 
Hybrid Reusable Computational Analytics Workflow
Management Framework.

\subsection{Reusable Computational Analytics}

{\em Reusable computational analytics} (RCA) focuses on  the creation of reusable programs, patterns, and services to conduct analytics tasks that are part of the scientific discovery process.  RCA service need varies widely and may include multi-scale hardware resources as well as multi-scale scientific applications. To utilize such services and their resources in a reusable way we need to have a mechanism to express them in an easy fashion that goes beyond just the definition in one programming language or framework, but allows the integration into many different programming languages and frameworks so that services that may be designed in one framework or language may be reusable in others.

\subsection{Reusable Multi-scale Algorithms}

In current scientific problems we encounter a rich set of applications that leverage a number of sophisticated methods that may require adaptations on multiple scales. The scales are influenced by their Domain size, accuracy and time requirement to solve them in sufficient manner. It is of advantage to provide reusable components that can be controlled by parameters to simplify reuse.

\subsection{Hybrid Cloud and Compute Resources and Serices}

As we deal with multi-scale algorithms not every analytics task needs to be conducted on a High Performance Computer (HPC). This is especially the case with the advent of GPUs in the desktop which we termed in the authors termed in past with {\em desktop supercomputing}. Also the availability of cloud computers and hyper-scale data centers play a significant role in today's analytics processes. This not only includes the use compute resources, but also services that are these days offered by cloud service providers. A well known example for this is natural language processing.

\subsection{Reusable and Adaptable HPC and Cloud Service Workflows}

High-performance computing (HPC) has been, for decades, a very important tool
for science. Scientific tasks can leverage the processing power of
a supercomputer so they can run at previously unobtainable high speeds
or utilize specialized hardware for acceleration that otherwise are not
available to the user. HPC can be used for analytic programs that
leverage machine learning applied to large data sets to, for example,
predict future values or to model current states. For such
high-complexity projects, there are often multiple complex programs that
may be running repeatedly in either competition or cooperation.
Leveraging computational GPUs, for instance, leads to several times higher
performance when applied to deep learning algorithms. With such
projects, program execution is submitted as a job to a typically remote
HPC center, where time is billed as node hours. Such projects must have
a service that lets the user manage and execute without supervision. We
have created a service that lets the user run jobs across multiple
platforms in a dynamic queue with visualization and data storage.

Similar aspects are available for cloud services that abstract the infrastructure needs and focus on the availability of services that can be integrated in concert with HPC, as well as the users local resources (for example a PC).


