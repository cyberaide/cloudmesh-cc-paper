\section{Contributing}\label{contributing}

\subsection{GitHub}\label{github}

The code is maintained on GitHub

\begin{itemize}
\tightlist
\item
  \href{https://github.com/cloudmesh/cloudmesh-cc}{Code}
\item
  \href{https://github.com/cloudmesh/cloudmesh-cc/issues}{Issues}
\item
  \href{https://github.com/cloudmesh/cloudmesh-cc/pulls}{Pull Requests}
\item
  \href{https://github.com/cloudmesh/cloudmesh-cc/actions}{Actions}
\item
  \href{https://github.com/cloudmesh/cloudmesh-cc/pulse}{Insights}
\end{itemize}

The code was developed by Gregor von Laszewski and J.P. Fleischer. Other
contributors to a pre alpha version are listed at

\begin{itemize}
\tightlist
\item
  \href{https://github.com/cloudmesh/cloudmesh-cc/graphs/contributors}{Contributors}
\end{itemize}

\subsection{Management}\label{management}

All contributions will be done under the apache license. The code will
be managed in open-source. Pull requests need to be made in the
repository. The main branch is the release branch. Contributions will
first be done in other branches, and once we agree that they need to be
integrated into the code, they will be merged into main. All new code
must have sufficient pytests. The pytests may need to be documented in
case special authentication is required.

We follow a
\href{https://docs.github.com/en/get-started/quickstart/github-flow}{typical
GitHub workflow} of:

\begin{itemize}
\tightlist
\item
  create a personal fork of this repo
\item
  create a branch
\item
  open a pull request
\item
  fix findings of various linters and checks
\item
  work through code review
\end{itemize}

For each pull request, the documentation is built and deployed to make
it easier to review the changes in the pull request.

\hypertarget{tests}{%
\subsection{Tests}\label{tests}}

Before creating a pull request it is important that the tests in the
test directory pass.

The tests are organized as follows:

\begin{itemize}
\tightlist
\item
  \texttt{pytest\ tests/test\_0??\_*} will run locally and only use
  ascii output
\item
  \texttt{pytest\ tests/test\_1??\_*} will run locally but also present
  graphs
\end{itemize}

in future we will likely change this and allow a test variable
\texttt{cms\ set\ test\_with\_graph=False/True} and if it is not
existent it is False. In case it is false the graohs will not be
displayed. but the test will be run. This change will also allow tests
with 1?? to be run on github workflows

\begin{itemize}
\tightlist
\item
  \texttt{pytest\ tests/test\_5??\_*} will run on remote machines and
  require at this time rivanna from UVA
\end{itemize}

in future this test will be modified so we can specify the remote user
and remote host \texttt{cms\ set\ test\_remote\_user=TBD}
\texttt{cms\ set\ test\_remote\_host=TBD}. If they do not exist they
will use the defualts from ssh config rivanna. verify if logic is ok.

\begin{itemize}
\tightlist
\item
  \texttt{pytest\ tests/test\_6??\_*} will test the rest service
\item
  \texttt{pytest\ tests/test\_9??\_*} will cleanup the tests
\end{itemize}
