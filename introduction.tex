\FILE{introduction.tex}

\section{What Is Cloudmesh cc?}\label{what-is-cloudmesh-cc}

Cloudmesh Compute Cluster (cc for short) is a repository of Python code
within the Cloudmesh suite that enables the creation of workflows. The
repository is compatible with Windows, macOS, and Linux.

Workflows are compilations of jobs to be run on nodes. These workflows
report information on the status of jobs as they are run, whether
locally or remotely. These jobs can be bash scripts, Python scripts,
Jupyter notebooks, or Slurm scripts. These types of jobs can be mixed
with one another in a single workflow.

The advantage that Cloudmesh cc provides is an interface to view the
progress of a workflow as it is being run in realtime. The workflow can
be monitored as a graph or a datatable. These interfaces report the
statuses of the jobs. Additionally, the order in which the jobs are run
can be specified, enabling prerequisite jobs and the segmentation of a
workflow.

Cloudmesh cc is bundled with pytests and examples that can be run on
your local machine or on HPC centers such as UVA's Rivanna supercomputer
(authentication is required for the latter). We encourage you to adapt
these tests to fit your HPC environment.
